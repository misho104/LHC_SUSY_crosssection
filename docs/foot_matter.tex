\section{Conclusion}\label{sec:conc}

This paper introduces a Python package ``\texttt{susy\_cross\_section}'' to handle cross-section grid tables provided by various groups in various format.
Users can install it via \texttt{PyPI} to load the grid tables, to get cross-section values by interpolations, and to validate the interpolating function, in an unified manner in their Python codes.
A terminal command ``\texttt{susy-xs}'' is provided, with which users can glance the grid data and cross-section values.
Several grid tables are pre-installed, which are originally calculated and distributed by NNLL-fast collaboration and LHC SUSY Cross Section Working Group.

This package is managed on \href{https://github.com}{GitHub}: \href{https://github.com/misho104/susy_cross_section}{\texttt{misho104/susy\_cross\_section}}.
Bug reports, comments, questions, and contribution are welcome on the website.


\section*{Acknowledgments}

(To be written.)

The author thanks %
%Christoph Borschensky,
%Anna Kulesza,
Kazuki Sakurai,
%...
for useful discussions.

The author acknowledges the works
by NNLL-fast collaborations
and
by LHC SUSY Cross Section Working Group.

The author credits PyPI as the distributor, GitHub as the repository host, TravisCI and CoverAlls as the host of continuous integration with coverage, and ReadTheDocs as the host of documentation, of this package.

The author (SI) is supported
by the MIUR-PRIN project 2015P5SBHT 003 ``Search for the Fundamental Laws and Constituents''
by INFN (to be clarified),...



